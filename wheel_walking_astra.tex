\documentclass[a4paper,twocolumn]{esapub2005} % European paper
\pagestyle{empty}

% introduce this option for the ESA publications style
\bibliographystyle{alpha}

\usepackage{graphicx} %for figures
\usepackage{subfig} %for subfigures
\usepackage[fleqn]{amsmath}    % need for subequations
\usepackage{multirow} % for complex tables
\usepackage{multicol} % for columns spliting
\usepackage{color} %for highlight text
\usepackage{booktabs}
\usepackage{url}
\usepackage{units}
\usepackage{floatrow}
\usepackage{float}
\restylefloat{table}

\title{First Experimental Investigations on Wheel-Walking for Improving Triple-Bogie Rover Locomotion Performances}

\author{Robin Nelen}
\author{Martin Zwick}
\author{Martin Azkarate}
\author{Tim Wiese}
\author{Javier Hidalgo-Carrio}
\author{Luc Joudrier}
\author{Pantelis Poulakis}
\author{Gianfranco Visentin}
\author{Michel Van Winnendael}
\affil{European Space Agency, ESA, Noordwijk, The Netherlands}
\affil{Technische Universit\"at M\"unchen, TUM, Munich, Germany}
\affil{Robotics Innovation Center, DFKI, Bremen, Germany}


\newcommand{\btx}{\textsc{Bib}\TeX}
\newcommand{\filename}{esapub}

\begin{document}

\keywords{\LaTeX; ESA; macros}

\maketitle

\section*{Abstract}
On November 2014 a test campaign was carried out in the Automation and Robotics
Laboratory of ESTEC to evaluate the performance of Wheel Walking manoeuvres in
different scenarios. This paper shows the experimental results obtained during
the campaign and the performance analysis made when comparing wheel walking
with standard rolling.


\section{Introduction}
Planetary rover missions to Mars such as NASA’s MER or MSL have shown a clear
decrease of the traversing capabilities while rolling across some particular
surface areas of Mars. Loose soil and fine dust terrain are scenarios where
rovers might experience the most significant loss of their tractive
performance. The extreme of these cases was encountered in the Spirit rover
which got permanently stuck on May 2009 while traversing the loose sandy area
of Troy {\color{red} [REF]}. Previous studies at JPL
{\color{red} [REF]} introduce the concept of the Effective Ground
Pressure (EGP) as a first order of approximation for an inversely proportional
indicator of the traversing capabilities of a rover. In relation to this,
ExoMars’ estimated EGP (considering an effectively larger wheel due to its
flexible properties) is about 2.5 times the one of MER and MSL.  Deployment
actuators of a triple-bogie rover locomotion platform could in principle be
used to perform wheelwalking manoeuvres.  How wheel walking could affect the
traversing capabilities of rovers when driving on the aforementioned
disadvantageous terrain types is a recurrent debate in planetary robotics
community.  This led a research group of the Automation and Robotics Lab of
ESTEC to carry out a testing campaign to evaluate the performance of wheel
walking motion. ExoTeR, the lab rover prototype that mimics the ExoMars
locomotion configuration (as design in 2007) in a scaled down version of it,
was used here for the test campaign. Three different scenarios were considered:
loose sand, slope and egress. Wheel walking outperformed standard rolling in
the three of them showing better traction in loose soil, increased gradeability
and more stability in egress sequences.  This paper briefly presents the ExoTeR
rover and the wheel walking concept in triple-bogie configurations and focuses
on the experimental results obtained for all three scenarios, comparing key
performance metrics between the wheel walking and rolling cases.


\section{The ExoMars Test Rover}


\section{Wheel Walking Implementation}

Global body commands of motion are commonly performed using a motion model.
Kinematics motion models have real-time capabilities and are inexpensive in
comparison with sophisticated wheel-dynamic simulation techniques.  The
wheel-walking evaluation presented in this paper uses a method which is able to
\textit{optimally}\footnotemark[2] combine the motion induced at each contact
point. The primary contribution of the model is fusing, in a unified framework,
desired body velocities to joint motion commands as a whole.  The
implementation of a complete motion model behaves more consistent and stable
than previous wheel-walking techniques. The model makes use of the
transformation approach~\cite{Tarokh2005} to accurately model 6-DoF kinematics.
It derives from the work in~\cite{Hidalgo-Carrio2014} to invert the Jacobian
formula from the odometry kinematics.

The model requires a minimum of two coordinate frames per kinematic chain:
a robot body frame ($B$) attached to the desired rover center and a
contact frame ($C_{il}$) defined as a single point of contact between the robot
and the ground. Those coordinate frames are related to each other by means of
the Jacobian matrix in the velocity domain. The matrix maps Cartesian to joint velocities and
relates the rover pose rates to joints and sensed rate quantities as:

\begin{equation}
    \left[\dot{x}_{B} ~ \dot{y}_{B} ~ \dot{z}_{B} ~ \dot{\phi}_{B} ~ \dot{\theta}_{B} ~ \dot{\psi}_{B} ~ \right]^T =
    J_{il} \left[\boldsymbol{\dot{q}} ~ \boldsymbol{\dot{\varepsilon}}_{il} \right]^T
\label{eq:wheeljacobian}
\end{equation}

where $\boldsymbol{\dot{q}}$ are the desire joint rates to command to the
actuators.  Different wheel-walking gaits are set by dynamically setting
constraints in the Jacobian. Each Jacobian defines the contribution of each
kinematic chain to the body motion allowing the analysis of each chain and
contact point to the resulting final velocity in the robot body.  Considering a
single contact angle $\dot{\varepsilon}_{il}$ the $J_{il}$ matrix size is $6
\times (n + 6)$ where $n$ corresponds to the DoF of the mechanism.  The
composite rover equations are obtained combining the Jacobian matrices for all
kinematics chains into a sparse matrix equation of appropriate dimensions. The
desired solution is obtained by numerically solving a system of equations using
Weighted Least-Squares.


\footnotetext[2]{Optimally here refers to the best estimated value from a
weighted least-squares perspective.}

\section{Experimental results}

\subsection{Loose sand}
\subsection{Slope Tests}
\subsection{Egress Maneuvers}
\subsection{Results Analysis}

\section{Conclusions}

\section{Future Work}


\vspace{-3 mm}

%%%%%%%%%%%%%%%%%%%%%%%%%%%%%%%%%%%%%%%%%%%%%%%%%%%%%%%%%%%%%%%%%%%%%%%%%%%%%%%%
% \footnotesize
\bibliographystyle{aa}
\bibliography{references}

% \end{small}
\end{document}
