\documentclass[a4paper,twocolumn]{esapub2005} % European paper
\pagestyle{empty}

% introduce this option for the ESA publications style
\bibliographystyle{alpha}

\usepackage{times}
\usepackage{natbib}
\usepackage{graphicx}

\title{First Experimental Investigations on Wheel-Walking for Improving Triple-Bogie Rover Locomotion Performances}

\author{Robin Nelen}
\author{Martin Zwick}
\author{Martin Azkarate}
\author{Tim Wiese}
\author{Javier Hidalgo-Carrio}
\author{Luc Joudrier}
\author{Pantelis Poulakis}
\author{Gianfranco Visentin}
\author{Michel Van Winnendael}
\affil{European Space Agency, ESA, Noordwijk, The Netherlands}
\affil{Technische Universit\"at M\"unchen, TUM, Munich, Germany}
\affil{Robotics Innovation Center, DFKI, Bremen, Germany}


\newcommand{\btx}{\textsc{Bib}\TeX}
\newcommand{\filename}{esapub}

\begin{document}

\keywords{\LaTeX; ESA; macros}

\maketitle

\section*{Abstract}
On November 2014 a test campaign was carried out in the Automation and Robotics
Laboratory of ESTEC to evaluate the performance of Wheel Walking manoeuvres in
different scenarios. This paper shows the experimental results obtained during
the campaign and the performance analysis made when comparing wheel walking
with standard rolling.


\section{Introduction}
Planetary rover missions to Mars such as NASA’s MER or MSL have shown a clear
decrease of the traversing capabilities while rolling across some particular
surface areas of Mars. Loose soil and fine dust terrain are scenarios where
rovers might experience the most significant loss of their tractive
performance. The extreme of these cases was encountered in the Spirit rover
which got permanently stuck on May 2009 while traversing the loose sandy area
of Troy REF] Previous studies at JPL REF introduce the concept of the Effective
Ground Pressure (EGP) as a first order of approximation for an inversely
proportional indicator of the traversing capabilities of a rover. In relation
to this, ExoMars’ estimated EGP (considering an effectively larger wheel due to
its flexible properties) is about 2.5 times the one of MER and MSL.  Deployment
actuators of a triple-bogie rover locomotion platform could in principle be
used to perform wheelwalking manoeuvres.  How wheel walking could affect the
traversing capabilities of rovers when driving on the aforementioned
disadvantageous terrain types is a recurrent debate in planetary robotics
community.  This led a research group of the Automation and Robotics Lab of
ESTEC to carry out a testing campaign to evaluate the performance of wheel
walking motion. ExoTeR, the lab rover prototype that mimics the ExoMars
locomotion configuration (as design in 2007) in a scaled down version of it,
was used here for the test campaign. Three different scenarios were considered:
loose sand, slope and egress. Wheel walking outperformed standard rolling in
the three of them showing better traction in loose soil, increased gradeability
and more stability in egress sequences.  This paper briefly presents the ExoTeR
rover and the wheel walking concept in triple-bogie configurations and focuses
on the experimental results obtained for all three scenarios, comparing key
performance metrics between the wheel walking and rolling cases.


\section{The ExoMars Test Rover}


\section{Wheel Walking Implementation}


\section{Experimental results}

\subsection{Loose sand}
\subsection{Slope Tests}
\subsection{Egress Maneuvers}
\subsection{Results Analysis}

\section{Conclusions}

\section{Future Work}

%\begin{thebibliography}{}
%\end{thebibliography}
\end{document}
